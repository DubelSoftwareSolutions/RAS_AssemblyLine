%Autorzy: Krzysztof Dąbek, Dymitr Choroszczak

%Praca inżynierska iRVision
%Ustawienia formatowania dokumentu 
%Autorzy: Krzysztof Dąbek

% Ustawienia kodowania 
% !TeX encoding = UTF-8

% Ustawienia sprawdzania poprawności pisowni
% !TeX spellcheck = pl_PL

% Określenie typu dokumentu i rozmiaru czcionki
\documentclass[12pt, a4paper]{article}

%Ustawienia strony i marginesów
\usepackage[a4paper, left=1.5cm, right=1.5cm, top=1.5cm, bottom=1.5cm, headsep=1.2cm]{geometry}

%Ustawienia języka dokumentu
%\usepackage[polish]{babel}
%\usepackage{polski}
\usepackage[utf8]{inputenc}
\usepackage{indentfirst}

%Ustawienia czcionki
\usepackage[T1]{fontenc}
\usepackage{xcolor}
\definecolor{textgray}{gray}{0.1}
\definecolor{framebg}{gray}{0.9}

%Ustawienia ramek
\usepackage{mdframed}
\definecolor{shadecolor}{RGB}{200,200,200}
\mdfdefinestyle{ReportFrameStyle}{
innerleftmargin=10,innerrightmargin=10,
topline=true,leftline=true,bottomline=true,rightline=true,
linecolor=black, shadow=false, linewidth=1,
middlelinewidth=0, innerlinewidth=0, outerlinewidth=0,
backgroundcolor=framebg}

%Pakiety obrazów
\usepackage{graphicx}
\usepackage{subfig}
\usepackage{rotating}
\usepackage{float}

%Pakiety tabel
\usepackage{supertabular}
\usepackage{array}
\usepackage{tabularx}
\usepackage{hhline}
\usepackage{longtable}
\usepackage{multicol}
\usepackage{colortbl}

%Pakiety import / eksport
\usepackage{url}
\usepackage{listings}
\usepackage{hyperref}

%Pakiety edytorskie
\usepackage{enumitem}

%Pakiety programisty
%\usepackage{forloop}
%\usepackage{calc}

\usepackage{mathtools}
%\DeclarePairedDelimiter\abs{\lvert}{\rvert}
%\usepackage{amssymb}


%Definicje grubszych kolumn
\newcolumntype{L}[1]{>{\raggedright\let\newline\\\arraybackslash\hspace{0pt}}m{#1}}
\newcolumntype{C}[1]{>{\centering\let\newline\\\arraybackslash\hspace{0pt}}m{#1}}
\newcolumntype{R}[1]{>{\raggedleft\let\newline\\\arraybackslash\hspace{0pt}}m{#1}}

%Zawijanie długich linii
\sloppy

%Resetowanie numerowania stron
\pagenumbering{gobble}
%Zmienne dokumentu
%Autorzy: Krzysztof Dąbek, Dymitr Choroszczak

% Zmienne wykorzystywane w programie
% Tytuł pracy
\newcommand{\ThesisTitle}{Event-based Control -- Project}

% Aspekt inżynierski
\newcommand{\ThesisAspect}{Task 5:    FMS controller}

% Autor pracy
\newcommand{\ThesisAuthor}{Dymitr Choroszczak (218627), Krzysztof Dąbek (218549)}

% Promotor pracy
\newcommand{\ThesisSupervisor}{Dr hab. inż. Elżbieta Roszkowska}

% Uczelnia
\newcommand{\UniversityName}{Politechnika Wrocławska}
\newcommand{\UniversityNameEng}{Wrocław University of Science and Technology}
\newcommand{\DeparamentName}{Faculty of Elektroncs}
\newcommand{\InstituteName}{Chair of Cybernetics and Robotics}
\newcommand{\Field}{Control Engineering and Robotics}
\newcommand{\FacoultyName}{Faculty of Electronics}
\newcommand{\Specialty}{Embedded robotics (AER)}
\newcommand{\IndexNumber}{218627}
\newcommand{\Logo}{\FigDir /logo_eng.png}

% Ścieżka do obrazów
\newcommand{\FigDir}{./figures}
%szczauka
\newcommand{\ra}{\rightarrow}

\renewcommand*{\figurename}{Figure.} 

% Makro do tworzenia strony tytułowej
\newcommand{\TitlePage}{
\begin{figure}
\centering
\includegraphics[height=0.4\textheight]{\Logo}
\end{figure}
\vspace{10cm}
\centering
\begin{Huge}
\textbf{\ThesisTitle} \\
\vspace{2cm}
\ThesisAspect \\
\vspace{3cm}
\end{Huge}
\flushleft 
\textbf{Author:} \ThesisAuthor \\
%\textbf{Index Number:} \IndexNumber \\
\textbf{Teacher:} \ThesisSupervisor \\
%\textbf{University:} \UniversityNameEng \\
\textbf{Faculty:} \FacoultyName \\
\textbf{Field of study} \Field \\
\textbf{Speciality:} \Specialty \\
\newpage
}


\begin{document}
\TitlePage

\section*{Introduction}
Current report contains solution of task 5 on project of Event-based Control cours. This task was done via using Python programming language and Snakes toolbox which belongs to it. 

The task consisted of developing controller for a flexible manufacturing cell consisting of
one transport robot and 5 machines. (fig. \ref{approach}). 
		\begin{figure}[!tp]
		\centering
		\includegraphics[scale=0.7]{figures/approach.png}
		\caption{Exemplary approach \label{approach}}
		\end{figure}
			
\section{Model}
\subsection{Main model}
Main system model with 5 robots is presented on figure \ref{approach}.
		
\subsection{Simulator}
Simulator was developed in order to manage simulation process. 
Simulator contains following instances:
\begin{itemize}
\item Net,
\item Transition Fired,
\item Transition Disabled, 
\item Process Operations, 
\item Operation Duration, 
\item Progress Timers, 
\item Started Processes, 
\item Number Of Parts 
\end{itemize}
which store manufacturing cell configuration and current state. Given class allows to initiate P/T net with all required resources and transitions.
					
		
\subsection{Controller}
The controller design, which proceeds permissions on successive transitions, and grants permissions to most significant ones. Permissions are granted first of all to transitions with highest priority. Priorities are stored in class in following order with growing priorities.

\begin{itemize}
\item Process Finish Priority,
\item Operation Finish Priority,
\item Operation Begin Priority,
\item Operation Prepare Priority
\end{itemize}
		
\section{Simulation}
The simulation results of designed model are presented on figure \ref{example_simulation}. Simulation time was established equal to 60s. Plot shows the states in which machines was in individual time points: free or busy.
		
		\begin{figure}[!tp]
		\centering
		\includegraphics[width=0.8\textwidth]{figures/example_simulation.png}
		\caption{Simulation \label{example_simulation}}
		\end{figure}			
		
Configuration of cell is showed on figure \ref{example_config}.	
				
		\begin{figure}[!tp]
		\centering
		\includegraphics[width=0.8\textwidth]{figures/example_config.png}
		\caption{Configuration \label{example_config}}
		\end{figure}			
		
\end{document}
